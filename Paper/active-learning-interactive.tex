\subsection{Active learning}
\label{active-learning-interactive}

A series of recent research works discuss and introduce a variety of model- and data-based methods to select the most informative samples to improve an NMT's performance.
\citet{zhang2018active} propose two active learning techniques for sentence selection, based on semantic similarity between the sentence embeddings and decoder probability as a token of translation quality, while \citet{peris2020active} explore the application of active learning methods in the interactive translation of large data streams. The latter work uses the attention mechanism of a neural machine translation system to select only those sentences worthy of human annotation for translation. 

In a more recent approach, \citet{zeng2019empirical} perform an empirical evaluation of different AL methods for Transformers \cite{vaswani2017attention}, analyzing data-driven approaches, that sample instances based only on two sets of labeled and unlabeled sentences, and model-driven methods, which additionally incorporate information from the current model into sampling. 
However their work notes the weak performance of Coverage Sampling compared to the other techniques, due to the complex nature of the state-of-the art multi-layer multi-head Natural Language Processing architectures, a phenomenon we are addressing with the current work.

A series of works explored the shortcomings of Active Learning approaches in Machine Translation. \cite{lowell2018practical} observe the performance drop issues arising in applying Active Learning across different models and tasks and discuss the methods' generalization constraints. Similarly, \citet{koshorek2019limits} examine the potential and limitations of an oracle policy in selecting samples to maximize the performance of a neural machine translation system in learning sentence semantic representations, concluding on a non-significant difference with a random sentence selection approach.
