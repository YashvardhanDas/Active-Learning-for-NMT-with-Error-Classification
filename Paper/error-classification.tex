\subsection{Error Classification in Machine Translation}
\label{ssec:related_errors}


There exists a significant body of work in the area of error classification of text for machine translation, and proposed methods can be primarily divided into manual or automatic. In the former category, approaches focus on describing detailed error typologies to be used in conducting a systematic manual error annotation \cite{fishel2011automatic,vilar2006error, lommel2014multidimensional, farrus2010linguistic, costa2015linguistically}. On the other hand,  automatic error classification tools \cite{zeman2011addicter, popovic2011hjerson,popovic2015poor} detect predefined error types, by measuring the alignment of the output of a MT system and its reference translation. Further work in this direction incorporates the idea that a certain sentence error can fit in more than one categories \cite{klubivcka2018quantitative, lommel2014assessing} and extends previous approaches to multi-label and multi-error automatic classification \citet{popovic2019automatic}. 

In the most recent approaches, error detection is examined as an application of neural sequence labeling methods, which aim to detect errors and predict their surrounding context \cite{rei2017semi}, and currently serve as the state of the art machine learning methods, utilizing powerful contextual word embeddings in an unsupervised setting \cite{bell2019context}, overcoming the disadvantages of prior methods that require large amounts of labeled data to perform efficiently.